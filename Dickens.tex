\documentclass[12pt]{article}
\begin{document}

\begin{titlepage}
\centerline{FOOD INSUFFICIENCY IN THE UPCOUNTRY REGIONS OF UGANDA\\}
\paragraph*{•}
\centerline{  Prepared by: KONGORO DICKENS 16/U/18805 216020616.\\}
\paragraph*{•}
\paragraph*{•}
  \begin{flushright}
  The Report,\\
  DATE: $February,8^{th},2018$.
  \end{flushright}
\date{\today}
\end{titlepage}
\tableofcontents
\newpage

\section{Abstract}
\paragraph{•}The aim of this report was to investigate why there is food insufficiency in the upcountry regions of Uganda. A survey on the methods in which food growth and consumption was conducted. The results indicate that
the majority of the people in upcountry regions of Uganda have poor methods of crop growing backed up with poor and indigenous methods of cultivation. The report concludes that scientific solutions should be implemented to help these people improve on there food production.
\section{Introduction}
\paragraph{•}There has been a massive increase in food insufficiency in the upcountry regions of Uganda over the past five years and there is every indication that this will continue if solutions are not effectively implemented. According to FAO (Food Agriculture Organization 2007) by 2020 almost 80$\%$ of people in leaving in Upcountry  regions will be have a single meal each day. FAO describes this phenomenon as ‘serious in the extreme, potentially undermining the the growth and development in a society’ (2007, p 167). Currently in Karamoja about 100 people die in six months.
Recently the representatives at the Ugandan parliament  complained about the ignorance and less care the government is showing them.  At present there is no policy regarding the measures that can help put things back to order.
\section{Methods}
\paragraph{•}This research was conducted by questionnaire and investigated the Eastern and Northern parts of the country. A total of 412 questionnaires were distributed with residents fortnightly giving there say.
The questionnaire used Lekert scales to assess social attitudes and provided open ended responses for additional comments.
Survey collection boxes were located in every sub-county headquarters for a four week period. No
personal information was collected; the survey was voluntary and anonymous. There was an 85$\%$ response rate to the questionnaire.
\section{Conclusion}
\paragraph{•}The government should invest more on improving the agriculture standards of these unfortunate Ugandans.
\section{Recommendations}
\paragraph{•}It is recommended that Ministry Of Agriculture connive with the Ministry of ICT and come up with a scientific solutions towards this problem. Like encouraging development of applications that will help detect crop diseases and correctly advice on the prescriptions required.  
Finally, the policy needs to apply to all the other parts of the nation.
\section{Methodology}
\ This method was majorly applied methodology because we are looking for the proper solution to a practical problem and that is to eliminate food insufficiency in the up-country regions in Uganda.
On the other hand it is analytical as it involves analyzing the already available information about this problem in Uganda.
Finally it is quantitative as it involves expressions in mathematical format.   
 
\end{document}